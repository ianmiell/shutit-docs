%
% API Documentation for ShutIt
% Module shutit_global
%
% Generated by epydoc 3.0.1
% [Tue Feb 17 12:53:50 2015]
%

%%%%%%%%%%%%%%%%%%%%%%%%%%%%%%%%%%%%%%%%%%%%%%%%%%%%%%%%%%%%%%%%%%%%%%%%%%%
%%                          Module Description                           %%
%%%%%%%%%%%%%%%%%%%%%%%%%%%%%%%%%%%%%%%%%%%%%%%%%%%%%%%%%%%%%%%%%%%%%%%%%%%

    \index{shutit\_global \textit{(module)}|(}
\section{Module shutit\_global}

    \label{shutit_global}
Contains all the core ShutIt methods and functionality.


%%%%%%%%%%%%%%%%%%%%%%%%%%%%%%%%%%%%%%%%%%%%%%%%%%%%%%%%%%%%%%%%%%%%%%%%%%%
%%                               Functions                               %%
%%%%%%%%%%%%%%%%%%%%%%%%%%%%%%%%%%%%%%%%%%%%%%%%%%%%%%%%%%%%%%%%%%%%%%%%%%%

  \subsection{Functions}

    \label{shutit_global:random_id}
    \index{shutit\_global \textit{(module)}!shutit\_global.random\_id \textit{(function)}}

    \vspace{0.5ex}

\hspace{.8\funcindent}\begin{boxedminipage}{\funcwidth}

    \raggedright \textbf{random\_id}(\textit{size}={\tt 8}, \textit{chars}={\tt \texttt{'}\texttt{abcdefghijklmnopqrstuvwxyzABCDEFGHIJKLMNOPQRSTUVWXYZ0123}\texttt{...}})

    \vspace{-1.5ex}

    \rule{\textwidth}{0.5\fboxrule}
\setlength{\parskip}{2ex}
    Generates a random string of given size from the given chars. size    -
    size of random string chars   - constituent pool of characters to draw 
    random characters from

\setlength{\parskip}{1ex}
    \end{boxedminipage}

    \label{shutit_global:init}
    \index{shutit\_global \textit{(module)}!shutit\_global.init \textit{(function)}}

    \vspace{0.5ex}

\hspace{.8\funcindent}\begin{boxedminipage}{\funcwidth}

    \raggedright \textbf{init}()

    \vspace{-1.5ex}

    \rule{\textwidth}{0.5\fboxrule}
\setlength{\parskip}{2ex}
    Initialize the shutit object. Called when imported.

\setlength{\parskip}{1ex}
    \end{boxedminipage}


%%%%%%%%%%%%%%%%%%%%%%%%%%%%%%%%%%%%%%%%%%%%%%%%%%%%%%%%%%%%%%%%%%%%%%%%%%%
%%                               Variables                               %%
%%%%%%%%%%%%%%%%%%%%%%%%%%%%%%%%%%%%%%%%%%%%%%%%%%%%%%%%%%%%%%%%%%%%%%%%%%%

  \subsection{Variables}

    \vspace{-1cm}
\hspace{\varindent}\begin{longtable}{|p{\varnamewidth}|p{\vardescrwidth}|l}
\cline{1-2}
\cline{1-2} \centering \textbf{Name} & \centering \textbf{Description}& \\
\cline{1-2}
\endhead\cline{1-2}\multicolumn{3}{r}{\small\textit{continued on next page}}\\\endfoot\cline{1-2}
\endlastfoot\raggedright s\-h\-u\-t\-i\-t\- & \raggedright \textbf{Value:} 
{\tt init()}&\\
\cline{1-2}
\raggedright \_\-\_\-p\-a\-c\-k\-a\-g\-e\-\_\-\_\- & \raggedright \textbf{Value:} 
{\tt None}&\\
\cline{1-2}
\raggedright c\-f\-g\- & \raggedright \textbf{Value:} 
{\tt \texttt{\{}\texttt{'}\texttt{action}\texttt{'}\texttt{: }\texttt{\{}\texttt{\}}\texttt{, }\texttt{'}\texttt{build}\texttt{'}\texttt{: }\texttt{\{}\texttt{'}\texttt{build\_id}\texttt{'}\texttt{: }\texttt{'}\texttt{rothko\_imiell\_14241}\texttt{...}}&\\
\cline{1-2}
\raggedright c\-w\-d\- & \raggedright \textbf{Value:} 
{\tt \texttt{'}\texttt{/space/git/shutit/docs/dev/html}\texttt{'}}&\\
\cline{1-2}
\raggedright p\-e\-x\-p\-e\-c\-t\-\_\-c\-h\-i\-l\-d\-r\-e\-n\- & \raggedright \textbf{Value:} 
{\tt \texttt{\{}\texttt{\}}}&\\
\cline{1-2}
\raggedright s\-h\-u\-t\-i\-t\-\_\-c\-o\-m\-m\-a\-n\-d\-\_\-h\-i\-s\-t\-o\-r\-y\- & \raggedright \textbf{Value:} 
{\tt \texttt{[}\texttt{]}}&\\
\cline{1-2}
\raggedright s\-h\-u\-t\-i\-t\-\_\-m\-a\-i\-n\-\_\-d\-i\-r\- & \raggedright \textbf{Value:} 
{\tt \texttt{'}\texttt{/space/git/shutit}\texttt{'}}&\\
\cline{1-2}
\raggedright s\-h\-u\-t\-i\-t\-\_\-m\-a\-p\- & \raggedright \textbf{Value:} 
{\tt \texttt{\{}\texttt{\}}}&\\
\cline{1-2}
\raggedright s\-h\-u\-t\-i\-t\-\_\-m\-o\-d\-u\-l\-e\-s\- & \raggedright \textbf{Value:} 
{\tt \texttt{set([}\texttt{])}}&\\
\cline{1-2}
\end{longtable}


%%%%%%%%%%%%%%%%%%%%%%%%%%%%%%%%%%%%%%%%%%%%%%%%%%%%%%%%%%%%%%%%%%%%%%%%%%%
%%                           Class Description                           %%
%%%%%%%%%%%%%%%%%%%%%%%%%%%%%%%%%%%%%%%%%%%%%%%%%%%%%%%%%%%%%%%%%%%%%%%%%%%

    \index{shutit\_global \textit{(module)}!shutit\_global.ShutIt \textit{(class)}|(}
\subsection{Class ShutIt}

    \label{shutit_global:ShutIt}
\begin{tabular}{cccccc}
% Line for object, linespec=[False]
\multicolumn{2}{r}{\settowidth{\BCL}{object}\multirow{2}{\BCL}{object}}
&&
  \\\cline{3-3}
  &&\multicolumn{1}{c|}{}
&&
  \\
&&\multicolumn{2}{l}{\textbf{shutit\_global.ShutIt}}
\end{tabular}

ShutIt build class. Represents an instance of a ShutIt build with 
associated config.


%%%%%%%%%%%%%%%%%%%%%%%%%%%%%%%%%%%%%%%%%%%%%%%%%%%%%%%%%%%%%%%%%%%%%%%%%%%
%%                                Methods                                %%
%%%%%%%%%%%%%%%%%%%%%%%%%%%%%%%%%%%%%%%%%%%%%%%%%%%%%%%%%%%%%%%%%%%%%%%%%%%

  \subsubsection{Methods}

    \vspace{0.5ex}

\hspace{.8\funcindent}\begin{boxedminipage}{\funcwidth}

    \raggedright \textbf{\_\_init\_\_}(\textit{self}, **\textit{kwargs})

    \vspace{-1.5ex}

    \rule{\textwidth}{0.5\fboxrule}
\setlength{\parskip}{2ex}
    Constructor. Sets up:

    \begin{itemize}
    \setlength{\parskip}{0.6ex}
      \item pexpect\_children   - pexpect objects representing shell 
        interactions

      \item shutit\_modules     - representation of loaded shutit modules

      \item shutit\_main\_dir    - directory in which shutit is located

      \item cfg                - dictionary of configuration of build

      \item cwd                - working directory of build

      \item shutit\_map         - maps module\_ids to module objects

    \end{itemize}

\setlength{\parskip}{1ex}
      Overrides: object.\_\_init\_\_

    \end{boxedminipage}

    \label{shutit_global:ShutIt:module_method_start}
    \index{shutit\_global \textit{(module)}!shutit\_global.ShutIt \textit{(class)}!shutit\_global.ShutIt.module\_method\_start \textit{(method)}}

    \vspace{0.5ex}

\hspace{.8\funcindent}\begin{boxedminipage}{\funcwidth}

    \raggedright \textbf{module\_method\_start}(\textit{self})

    \vspace{-1.5ex}

    \rule{\textwidth}{0.5\fboxrule}
\setlength{\parskip}{2ex}
    Gets called automatically by the metaclass decorator in shutit\_module 
    when a module method is called. This allows setting defaults for the 
    'scope' of a method.

\setlength{\parskip}{1ex}
    \end{boxedminipage}

    \label{shutit_global:ShutIt:module_method_end}
    \index{shutit\_global \textit{(module)}!shutit\_global.ShutIt \textit{(class)}!shutit\_global.ShutIt.module\_method\_end \textit{(method)}}

    \vspace{0.5ex}

\hspace{.8\funcindent}\begin{boxedminipage}{\funcwidth}

    \raggedright \textbf{module\_method\_end}(\textit{self})

    \vspace{-1.5ex}

    \rule{\textwidth}{0.5\fboxrule}
\setlength{\parskip}{2ex}
    Gets called automatically by the metaclass decorator in shutit\_module 
    when a module method is finished. This allows setting defaults for the 
    'scope' of a method.

\setlength{\parskip}{1ex}
    \end{boxedminipage}

    \label{shutit_global:ShutIt:get_default_child}
    \index{shutit\_global \textit{(module)}!shutit\_global.ShutIt \textit{(class)}!shutit\_global.ShutIt.get\_default\_child \textit{(method)}}

    \vspace{0.5ex}

\hspace{.8\funcindent}\begin{boxedminipage}{\funcwidth}

    \raggedright \textbf{get\_default\_child}(\textit{self})

    \vspace{-1.5ex}

    \rule{\textwidth}{0.5\fboxrule}
\setlength{\parskip}{2ex}
    Returns the currently-set default pexpect child.

\setlength{\parskip}{1ex}
    \end{boxedminipage}

    \label{shutit_global:ShutIt:get_default_expect}
    \index{shutit\_global \textit{(module)}!shutit\_global.ShutIt \textit{(class)}!shutit\_global.ShutIt.get\_default\_expect \textit{(method)}}

    \vspace{0.5ex}

\hspace{.8\funcindent}\begin{boxedminipage}{\funcwidth}

    \raggedright \textbf{get\_default\_expect}(\textit{self})

    \vspace{-1.5ex}

    \rule{\textwidth}{0.5\fboxrule}
\setlength{\parskip}{2ex}
    Returns the currently-set default pexpect string (usually a prompt).

\setlength{\parskip}{1ex}
    \end{boxedminipage}

    \label{shutit_global:ShutIt:get_default_check_exit}
    \index{shutit\_global \textit{(module)}!shutit\_global.ShutIt \textit{(class)}!shutit\_global.ShutIt.get\_default\_check\_exit \textit{(method)}}

    \vspace{0.5ex}

\hspace{.8\funcindent}\begin{boxedminipage}{\funcwidth}

    \raggedright \textbf{get\_default\_check\_exit}(\textit{self})

    \vspace{-1.5ex}

    \rule{\textwidth}{0.5\fboxrule}
\setlength{\parskip}{2ex}
    Returns default value of check\_exit. See send method.

\setlength{\parskip}{1ex}
    \end{boxedminipage}

    \label{shutit_global:ShutIt:set_default_child}
    \index{shutit\_global \textit{(module)}!shutit\_global.ShutIt \textit{(class)}!shutit\_global.ShutIt.set\_default\_child \textit{(method)}}

    \vspace{0.5ex}

\hspace{.8\funcindent}\begin{boxedminipage}{\funcwidth}

    \raggedright \textbf{set\_default\_child}(\textit{self}, \textit{child})

    \vspace{-1.5ex}

    \rule{\textwidth}{0.5\fboxrule}
\setlength{\parskip}{2ex}
    Sets the default pexpect child.

\setlength{\parskip}{1ex}
    \end{boxedminipage}

    \label{shutit_global:ShutIt:set_default_expect}
    \index{shutit\_global \textit{(module)}!shutit\_global.ShutIt \textit{(class)}!shutit\_global.ShutIt.set\_default\_expect \textit{(method)}}

    \vspace{0.5ex}

\hspace{.8\funcindent}\begin{boxedminipage}{\funcwidth}

    \raggedright \textbf{set\_default\_expect}(\textit{self}, \textit{expect}={\tt None}, \textit{check\_exit}={\tt True})

    \vspace{-1.5ex}

    \rule{\textwidth}{0.5\fboxrule}
\setlength{\parskip}{2ex}
    Sets the default pexpect string (usually a prompt). Defaults to the 
    configured root\_prompt if no argument is passed.

\setlength{\parskip}{1ex}
    \end{boxedminipage}

    \label{shutit_global:ShutIt:fail}
    \index{shutit\_global \textit{(module)}!shutit\_global.ShutIt \textit{(class)}!shutit\_global.ShutIt.fail \textit{(method)}}

    \vspace{0.5ex}

\hspace{.8\funcindent}\begin{boxedminipage}{\funcwidth}

    \raggedright \textbf{fail}(\textit{self}, \textit{msg}, \textit{child}={\tt None}, \textit{throw\_exception}={\tt True})

    \vspace{-1.5ex}

    \rule{\textwidth}{0.5\fboxrule}
\setlength{\parskip}{2ex}
    Handles a failure, pausing if a pexpect child object is passed in.

\setlength{\parskip}{1ex}
    \end{boxedminipage}

    \label{shutit_global:ShutIt:log}
    \index{shutit\_global \textit{(module)}!shutit\_global.ShutIt \textit{(class)}!shutit\_global.ShutIt.log \textit{(method)}}

    \vspace{0.5ex}

\hspace{.8\funcindent}\begin{boxedminipage}{\funcwidth}

    \raggedright \textbf{log}(\textit{self}, \textit{msg}, \textit{code}={\tt None}, \textit{pause}={\tt 0}, \textit{prefix}={\tt True}, \textit{force\_stdout}={\tt False}, \textit{add\_final\_message}={\tt False})

    \vspace{-1.5ex}

    \rule{\textwidth}{0.5\fboxrule}
\setlength{\parskip}{2ex}
\begin{alltt}
Logging function.

- code         - Colour code for logging. Ignored if we are in serve mode.
- pause        - Length of time to pause after logging (default: 0)
- prefix       - Whether to output logging prefix (LOG: {\textless}time{\textgreater}) (default: True)
- force\_stdout - If we are not in debug, put this in stdout anyway (default: False)
- add\_final\_message - Add this log line to the final message (report\_final\_message)
\end{alltt}

\setlength{\parskip}{1ex}
    \end{boxedminipage}

    \label{shutit_global:ShutIt:multisend}
    \index{shutit\_global \textit{(module)}!shutit\_global.ShutIt \textit{(class)}!shutit\_global.ShutIt.multisend \textit{(method)}}

    \vspace{0.5ex}

\hspace{.8\funcindent}\begin{boxedminipage}{\funcwidth}

    \raggedright \textbf{multisend}(\textit{self}, \textit{send}, \textit{send\_dict}, \textit{expect}={\tt None}, \textit{child}={\tt None}, \textit{timeout}={\tt 3600}, \textit{check\_exit}={\tt None}, \textit{fail\_on\_empty\_before}={\tt True}, \textit{record\_command}={\tt True}, \textit{exit\_values}={\tt None}, \textit{echo}={\tt None})

    \vspace{-1.5ex}

    \rule{\textwidth}{0.5\fboxrule}
\setlength{\parskip}{2ex}
    Multisend. Same as send, except it takes multiple sends and expects in 
    a dict that are processed while waiting for the end "expect" argument 
    supplied.

    Arguments as per send(), except:

    \begin{itemize}
    \setlength{\parskip}{0.6ex}
      \item send\_dict - dict of sends and expects, eg: \{'interim 
        prompt:','some input','other prompt','some other input'\}

      \item expect - final expect we want to see. defaults to 
        child.get\_default\_expect()

    \end{itemize}

\setlength{\parskip}{1ex}
    \end{boxedminipage}

    \label{shutit_global:ShutIt:send}
    \index{shutit\_global \textit{(module)}!shutit\_global.ShutIt \textit{(class)}!shutit\_global.ShutIt.send \textit{(method)}}

    \vspace{0.5ex}

\hspace{.8\funcindent}\begin{boxedminipage}{\funcwidth}

    \raggedright \textbf{send}(\textit{self}, \textit{send}, \textit{expect}={\tt None}, \textit{child}={\tt None}, \textit{timeout}={\tt 3600}, \textit{check\_exit}={\tt None}, \textit{fail\_on\_empty\_before}={\tt True}, \textit{record\_command}={\tt True}, \textit{exit\_values}={\tt None}, \textit{echo}={\tt False}, \textit{retry}={\tt 3})

    \vspace{-1.5ex}

    \rule{\textwidth}{0.5\fboxrule}
\setlength{\parskip}{2ex}
    Send string as a shell command, and wait until the expected output is 
    seen (either a string or any from a list of strings) before returning. 
    The expected string will default to the currently-set default expected 
    string (see get\_default\_expect)

    Returns the pexpect return value (ie which expected string in the list 
    matched)

    Arguments:

    \begin{itemize}
    \setlength{\parskip}{0.6ex}
      \item child                      - pexpect child to issue command to.

      \item send                       - String to send, ie the command being 
        issued. If set to None, we consume up to the expect string, which 
        is useful if we just matched output that came before a standard 
        command that returns to the prompt.

      \item expect                     - String that we expect to see in the 
        output. Usually a prompt. Defaults to currently-set expect string 
        (see set\_default\_expect)

      \item timeout                    - Timeout on response (default=3600 
        seconds).

      \item check\_exit                 - Whether to check the shell exit code 
        of the passed-in command.  If the exit value was non-zero an error 
        is thrown. (default=None, which takes the currently-configured 
        check\_exit value) See also fail\_on\_empty\_before.

      \item fail\_on\_empty\_before       - If debug is set, fail on empty 
        match output string (default=True) If this is set to False, then we
        don't check the exit value of the command.

      \item record\_command             - Whether to record the command for 
        output at end (default=True). As a safety measure, if the command 
        matches any 'password's then we don't record it.

      \item exit\_values                - Array of acceptable exit values as 
        strings (default ['0'])

      \item echo                       - Whether to suppress any logging output
        from pexpect to the terminal or not. We don't record the command if
        this is set to False unless record\_command is explicitly passed in
        as True.

      \item retry                      - Number of times to retry the command 
        if the first attempt doesn't work. Useful if going to the network.

    \end{itemize}

\setlength{\parskip}{1ex}
    \end{boxedminipage}

    \label{shutit_global:ShutIt:send}
    \index{shutit\_global \textit{(module)}!shutit\_global.ShutIt \textit{(class)}!shutit\_global.ShutIt.send \textit{(method)}}

    \vspace{0.5ex}

\hspace{.8\funcindent}\begin{boxedminipage}{\funcwidth}

    \raggedright \textbf{send\_and\_expect}(\textit{self}, \textit{send}, \textit{expect}={\tt None}, \textit{child}={\tt None}, \textit{timeout}={\tt 3600}, \textit{check\_exit}={\tt None}, \textit{fail\_on\_empty\_before}={\tt True}, \textit{record\_command}={\tt True}, \textit{exit\_values}={\tt None}, \textit{echo}={\tt False}, \textit{retry}={\tt 3})

    \vspace{-1.5ex}

    \rule{\textwidth}{0.5\fboxrule}
\setlength{\parskip}{2ex}
    Send string as a shell command, and wait until the expected output is 
    seen (either a string or any from a list of strings) before returning. 
    The expected string will default to the currently-set default expected 
    string (see get\_default\_expect)

    Returns the pexpect return value (ie which expected string in the list 
    matched)

    Arguments:

    \begin{itemize}
    \setlength{\parskip}{0.6ex}
      \item child                      - pexpect child to issue command to.

      \item send                       - String to send, ie the command being 
        issued. If set to None, we consume up to the expect string, which 
        is useful if we just matched output that came before a standard 
        command that returns to the prompt.

      \item expect                     - String that we expect to see in the 
        output. Usually a prompt. Defaults to currently-set expect string 
        (see set\_default\_expect)

      \item timeout                    - Timeout on response (default=3600 
        seconds).

      \item check\_exit                 - Whether to check the shell exit code 
        of the passed-in command.  If the exit value was non-zero an error 
        is thrown. (default=None, which takes the currently-configured 
        check\_exit value) See also fail\_on\_empty\_before.

      \item fail\_on\_empty\_before       - If debug is set, fail on empty 
        match output string (default=True) If this is set to False, then we
        don't check the exit value of the command.

      \item record\_command             - Whether to record the command for 
        output at end (default=True). As a safety measure, if the command 
        matches any 'password's then we don't record it.

      \item exit\_values                - Array of acceptable exit values as 
        strings (default ['0'])

      \item echo                       - Whether to suppress any logging output
        from pexpect to the terminal or not. We don't record the command if
        this is set to False unless record\_command is explicitly passed in
        as True.

      \item retry                      - Number of times to retry the command 
        if the first attempt doesn't work. Useful if going to the network.

    \end{itemize}

\setlength{\parskip}{1ex}
    \end{boxedminipage}

    \label{shutit_global:ShutIt:run_script}
    \index{shutit\_global \textit{(module)}!shutit\_global.ShutIt \textit{(class)}!shutit\_global.ShutIt.run\_script \textit{(method)}}

    \vspace{0.5ex}

\hspace{.8\funcindent}\begin{boxedminipage}{\funcwidth}

    \raggedright \textbf{run\_script}(\textit{self}, \textit{script}, \textit{expect}={\tt None}, \textit{child}={\tt None}, \textit{in\_shell}={\tt True})

    \vspace{-1.5ex}

    \rule{\textwidth}{0.5\fboxrule}
\setlength{\parskip}{2ex}
\begin{alltt}
Run the passed-in string as a script on the target's command line.

- script   - String representing the script. It will be de-indented
                 and stripped before being run.
- expect   - See send()
- child    - See send()
- in\_shell - Indicate whether we are in a shell or not.
\end{alltt}

\setlength{\parskip}{1ex}
    \end{boxedminipage}

    \label{shutit_global:ShutIt:send_file}
    \index{shutit\_global \textit{(module)}!shutit\_global.ShutIt \textit{(class)}!shutit\_global.ShutIt.send\_file \textit{(method)}}

    \vspace{0.5ex}

\hspace{.8\funcindent}\begin{boxedminipage}{\funcwidth}

    \raggedright \textbf{send\_file}(\textit{self}, \textit{path}, \textit{contents}, \textit{expect}={\tt None}, \textit{child}={\tt None}, \textit{log}={\tt True})

    \vspace{-1.5ex}

    \rule{\textwidth}{0.5\fboxrule}
\setlength{\parskip}{2ex}
    Sends the passed-in string as a file to the passed-in path on the 
    target.

    \begin{itemize}
    \setlength{\parskip}{0.6ex}
      \item path     - Target location of file on target.

      \item contents - Contents of file as a string. See log.

      \item expect   - See send()

      \item child    - See send()

      \item log      - Log the file contents if in debug.

    \end{itemize}

\setlength{\parskip}{1ex}
    \end{boxedminipage}

    \label{shutit_global:ShutIt:chdir}
    \index{shutit\_global \textit{(module)}!shutit\_global.ShutIt \textit{(class)}!shutit\_global.ShutIt.chdir \textit{(method)}}

    \vspace{0.5ex}

\hspace{.8\funcindent}\begin{boxedminipage}{\funcwidth}

    \raggedright \textbf{chdir}(\textit{self}, \textit{path}, \textit{expect}={\tt None}, \textit{child}={\tt None}, \textit{timeout}={\tt 3600}, \textit{log}={\tt True})

    \vspace{-1.5ex}

    \rule{\textwidth}{0.5\fboxrule}
\setlength{\parskip}{2ex}
    How to change directory will depend on whether we are in delivery mode 
    bash or docker.

\setlength{\parskip}{1ex}
    \end{boxedminipage}

    \label{shutit_global:ShutIt:send_host_file}
    \index{shutit\_global \textit{(module)}!shutit\_global.ShutIt \textit{(class)}!shutit\_global.ShutIt.send\_host\_file \textit{(method)}}

    \vspace{0.5ex}

\hspace{.8\funcindent}\begin{boxedminipage}{\funcwidth}

    \raggedright \textbf{send\_host\_file}(\textit{self}, \textit{path}, \textit{hostfilepath}, \textit{expect}={\tt None}, \textit{child}={\tt None}, \textit{timeout}={\tt 3600}, \textit{log}={\tt True})

    \vspace{-1.5ex}

    \rule{\textwidth}{0.5\fboxrule}
\setlength{\parskip}{2ex}
\begin{alltt}
Send file from host machine to given path

- path         - Path to send file to.
- hostfilepath - Path to file from host to send to target.
- expect       - See send()
- child        - See send()
- log          - arg to pass to send\_file (default True)
\end{alltt}

\setlength{\parskip}{1ex}
    \end{boxedminipage}

    \label{shutit_global:ShutIt:send_host_dir}
    \index{shutit\_global \textit{(module)}!shutit\_global.ShutIt \textit{(class)}!shutit\_global.ShutIt.send\_host\_dir \textit{(method)}}

    \vspace{0.5ex}

\hspace{.8\funcindent}\begin{boxedminipage}{\funcwidth}

    \raggedright \textbf{send\_host\_dir}(\textit{self}, \textit{path}, \textit{hostfilepath}, \textit{expect}={\tt None}, \textit{child}={\tt None}, \textit{log}={\tt True})

    \vspace{-1.5ex}

    \rule{\textwidth}{0.5\fboxrule}
\setlength{\parskip}{2ex}
    Send directory and all contents recursively from host machine to given 
    path.  It will automatically make directories on the target.

    \begin{itemize}
    \setlength{\parskip}{0.6ex}
      \item path         - Path to send directory to

      \item hostfilepath - Path to file from host to send to target

      \item expect       - See send()

      \item child        - See send()

      \item log          - Arg to pass to send\_file (default True)

    \end{itemize}

\setlength{\parskip}{1ex}
    \end{boxedminipage}

    \label{shutit_global:ShutIt:host_file_exists}
    \index{shutit\_global \textit{(module)}!shutit\_global.ShutIt \textit{(class)}!shutit\_global.ShutIt.host\_file\_exists \textit{(method)}}

    \vspace{0.5ex}

\hspace{.8\funcindent}\begin{boxedminipage}{\funcwidth}

    \raggedright \textbf{host\_file\_exists}(\textit{self}, \textit{filename}, \textit{directory}={\tt False})

    \vspace{-1.5ex}

    \rule{\textwidth}{0.5\fboxrule}
\setlength{\parskip}{2ex}
\begin{alltt}
Return True if file exists on the host, else false

- filename     - Filename to determine the existence of.
- directory    - Indicate that the file expected is a directory.
\end{alltt}

\setlength{\parskip}{1ex}
    \end{boxedminipage}

    \label{shutit_global:ShutIt:file_exists}
    \index{shutit\_global \textit{(module)}!shutit\_global.ShutIt \textit{(class)}!shutit\_global.ShutIt.file\_exists \textit{(method)}}

    \vspace{0.5ex}

\hspace{.8\funcindent}\begin{boxedminipage}{\funcwidth}

    \raggedright \textbf{file\_exists}(\textit{self}, \textit{filename}, \textit{expect}={\tt None}, \textit{child}={\tt None}, \textit{directory}={\tt False})

    \vspace{-1.5ex}

    \rule{\textwidth}{0.5\fboxrule}
\setlength{\parskip}{2ex}
\begin{alltt}
Return True if file exists on the target host, else False

- filename     - Filename to determine the existence of.
- expect       - See send()
- child        - See send()
- directory    - Indicate that the file is a directory.
\end{alltt}

\setlength{\parskip}{1ex}
    \end{boxedminipage}

    \label{shutit_global:ShutIt:get_file_perms}
    \index{shutit\_global \textit{(module)}!shutit\_global.ShutIt \textit{(class)}!shutit\_global.ShutIt.get\_file\_perms \textit{(method)}}

    \vspace{0.5ex}

\hspace{.8\funcindent}\begin{boxedminipage}{\funcwidth}

    \raggedright \textbf{get\_file\_perms}(\textit{self}, \textit{filename}, \textit{expect}={\tt None}, \textit{child}={\tt None})

    \vspace{-1.5ex}

    \rule{\textwidth}{0.5\fboxrule}
\setlength{\parskip}{2ex}
    Returns the permissions of the file on the target as an octal string 
    triplet.

    \begin{itemize}
    \setlength{\parskip}{0.6ex}
      \item filename  - Filename to get permissions of.

      \item expect    - See send()

      \item child     - See send()

    \end{itemize}

\setlength{\parskip}{1ex}
    \end{boxedminipage}

    \label{shutit_global:ShutIt:remove_line_from_file}
    \index{shutit\_global \textit{(module)}!shutit\_global.ShutIt \textit{(class)}!shutit\_global.ShutIt.remove\_line\_from\_file \textit{(method)}}

    \vspace{0.5ex}

\hspace{.8\funcindent}\begin{boxedminipage}{\funcwidth}

    \raggedright \textbf{remove\_line\_from\_file}(\textit{self}, \textit{line}, \textit{filename}, \textit{expect}={\tt None}, \textit{child}={\tt None}, \textit{match\_regexp}={\tt None}, \textit{literal}={\tt False})

    \vspace{-1.5ex}

    \rule{\textwidth}{0.5\fboxrule}
\setlength{\parskip}{2ex}
    Removes line from file, if it exists. Must be exactly the line passed 
    in to match. Returns True if there were no problems, False if there 
    were.

    \begin{itemize}
    \setlength{\parskip}{0.6ex}
      \item line         - Line to add.

      \item filename     - Filename to add it to.

      \item expect       - See send()

      \item child        - See send()

      \item match\_regexp - If supplied, a regexp to look for in the file 
        instead of the line itself, handy if the line has awkward 
        characters in it.

      \item literal      - If true, then simply grep for the exact string 
        without bash interpretation.

    \end{itemize}

\setlength{\parskip}{1ex}
    \end{boxedminipage}

    \label{shutit_global:ShutIt:add_line_to_file}
    \index{shutit\_global \textit{(module)}!shutit\_global.ShutIt \textit{(class)}!shutit\_global.ShutIt.add\_line\_to\_file \textit{(method)}}

    \vspace{0.5ex}

\hspace{.8\funcindent}\begin{boxedminipage}{\funcwidth}

    \raggedright \textbf{add\_line\_to\_file}(\textit{self}, \textit{line}, \textit{filename}, \textit{expect}={\tt None}, \textit{child}={\tt None}, \textit{match\_regexp}={\tt None}, \textit{force}={\tt True}, \textit{literal}={\tt False})

    \vspace{-1.5ex}

    \rule{\textwidth}{0.5\fboxrule}
\setlength{\parskip}{2ex}
    Adds line to file if it doesn't exist (unless Force is set, which it is
    by default). Creates the file if it doesn't exist. Must be exactly the 
    line passed in to match. Returns True if line added, False if not. If 
    you have a lot of non-unique lines to add, it's a good idea to have a 
    sentinel value to add first, and then if that returns true, force the 
    remainder.

    \begin{itemize}
    \setlength{\parskip}{0.6ex}
      \item line         - Line to add.

      \item filename     - Filename to add it to.

      \item expect       - See send()

      \item child        - See send()

      \item match\_regexp - If supplied, a regexp to look for in the file 
        instead of the line itself, handy if the line has awkward 
        characters in it.

      \item force        - Always write the line to the file.

      \item literal      - If true, then simply grep for the exact string 
        without bash interpretation.

    \end{itemize}

\setlength{\parskip}{1ex}
    \end{boxedminipage}

    \label{shutit_global:ShutIt:add_to_bashrc}
    \index{shutit\_global \textit{(module)}!shutit\_global.ShutIt \textit{(class)}!shutit\_global.ShutIt.add\_to\_bashrc \textit{(method)}}

    \vspace{0.5ex}

\hspace{.8\funcindent}\begin{boxedminipage}{\funcwidth}

    \raggedright \textbf{add\_to\_bashrc}(\textit{self}, \textit{line}, \textit{expect}={\tt None}, \textit{child}={\tt None}, \textit{match\_regexp}={\tt None})

    \vspace{-1.5ex}

    \rule{\textwidth}{0.5\fboxrule}
\setlength{\parskip}{2ex}
    Takes care of adding a line to everyone's bashrc (/etc/bash.bashrc, 
    /etc/profile).

    \begin{itemize}
    \setlength{\parskip}{0.6ex}
      \item line   - Line to add.

      \item expect - See send()

      \item child  - See send()

    \end{itemize}

\setlength{\parskip}{1ex}
    \end{boxedminipage}

    \label{shutit_global:ShutIt:get_url}
    \index{shutit\_global \textit{(module)}!shutit\_global.ShutIt \textit{(class)}!shutit\_global.ShutIt.get\_url \textit{(method)}}

    \vspace{0.5ex}

\hspace{.8\funcindent}\begin{boxedminipage}{\funcwidth}

    \raggedright \textbf{get\_url}(\textit{self}, \textit{filename}, \textit{locations}, \textit{command}={\tt \texttt{'}\texttt{wget}\texttt{'}}, \textit{expect}={\tt None}, \textit{child}={\tt None}, \textit{timeout}={\tt 3600}, \textit{fail\_on\_empty\_before}={\tt True}, \textit{record\_command}={\tt True}, \textit{exit\_values}={\tt None}, \textit{echo}={\tt False}, \textit{retry}={\tt 3})

    \vspace{-1.5ex}

    \rule{\textwidth}{0.5\fboxrule}
\setlength{\parskip}{2ex}
    Handles the getting of a url for you. filename is filename, eg ajar.jar
    locations is a list of mirrors, eg 
    get\_util('somejar.jar',['ftp://loc.org','http://anotherloc.com/jars'])

\setlength{\parskip}{1ex}
    \end{boxedminipage}

    \label{shutit_global:ShutIt:user_exists}
    \index{shutit\_global \textit{(module)}!shutit\_global.ShutIt \textit{(class)}!shutit\_global.ShutIt.user\_exists \textit{(method)}}

    \vspace{0.5ex}

\hspace{.8\funcindent}\begin{boxedminipage}{\funcwidth}

    \raggedright \textbf{user\_exists}(\textit{self}, \textit{user}, \textit{expect}={\tt None}, \textit{child}={\tt None})

    \vspace{-1.5ex}

    \rule{\textwidth}{0.5\fboxrule}
\setlength{\parskip}{2ex}
\begin{alltt}
Returns true if the specified username exists.

- user   - username to check for
- expect - See send()
- child  - See send()
\end{alltt}

\setlength{\parskip}{1ex}
    \end{boxedminipage}

    \label{shutit_global:ShutIt:package_installed}
    \index{shutit\_global \textit{(module)}!shutit\_global.ShutIt \textit{(class)}!shutit\_global.ShutIt.package\_installed \textit{(method)}}

    \vspace{0.5ex}

\hspace{.8\funcindent}\begin{boxedminipage}{\funcwidth}

    \raggedright \textbf{package\_installed}(\textit{self}, \textit{package}, \textit{expect}={\tt None}, \textit{child}={\tt None})

    \vspace{-1.5ex}

    \rule{\textwidth}{0.5\fboxrule}
\setlength{\parskip}{2ex}
\begin{alltt}
Returns True if we can be sure the package is installed.

- package - Package as a string, eg 'wget'.
- expect  - See send()
- child   - See send()
\end{alltt}

\setlength{\parskip}{1ex}
    \end{boxedminipage}

    \label{shutit_global:ShutIt:is_shutit_installed}
    \index{shutit\_global \textit{(module)}!shutit\_global.ShutIt \textit{(class)}!shutit\_global.ShutIt.is\_shutit\_installed \textit{(method)}}

    \vspace{0.5ex}

\hspace{.8\funcindent}\begin{boxedminipage}{\funcwidth}

    \raggedright \textbf{is\_shutit\_installed}(\textit{self}, \textit{module\_id})

    \vspace{-1.5ex}

    \rule{\textwidth}{0.5\fboxrule}
\setlength{\parskip}{2ex}
    Helper proc to determine whether shutit has installed already here by 
    placing a file in the db.

\setlength{\parskip}{1ex}
    \end{boxedminipage}

    \label{shutit_global:ShutIt:ls}
    \index{shutit\_global \textit{(module)}!shutit\_global.ShutIt \textit{(class)}!shutit\_global.ShutIt.ls \textit{(method)}}

    \vspace{0.5ex}

\hspace{.8\funcindent}\begin{boxedminipage}{\funcwidth}

    \raggedright \textbf{ls}(\textit{self}, \textit{directory})

    \vspace{-1.5ex}

    \rule{\textwidth}{0.5\fboxrule}
\setlength{\parskip}{2ex}
    Helper proc to list files in a directory

    Returns list of files.

    directory - directory to list

\setlength{\parskip}{1ex}
    \end{boxedminipage}

    \label{shutit_global:ShutIt:mount_tmp}
    \index{shutit\_global \textit{(module)}!shutit\_global.ShutIt \textit{(class)}!shutit\_global.ShutIt.mount\_tmp \textit{(method)}}

    \vspace{0.5ex}

\hspace{.8\funcindent}\begin{boxedminipage}{\funcwidth}

    \raggedright \textbf{mount\_tmp}(\textit{self})

    \vspace{-1.5ex}

    \rule{\textwidth}{0.5\fboxrule}
\setlength{\parskip}{2ex}
    mount a temporary file system as a workaround for the AUFS /tmp issues 
    not necessary if running devicemapper

\setlength{\parskip}{1ex}
    \end{boxedminipage}

    \label{shutit_global:ShutIt:get_file}
    \index{shutit\_global \textit{(module)}!shutit\_global.ShutIt \textit{(class)}!shutit\_global.ShutIt.get\_file \textit{(method)}}

    \vspace{0.5ex}

\hspace{.8\funcindent}\begin{boxedminipage}{\funcwidth}

    \raggedright \textbf{get\_file}(\textit{self}, \textit{target\_path}, \textit{host\_path})

    \vspace{-1.5ex}

    \rule{\textwidth}{0.5\fboxrule}
\setlength{\parskip}{2ex}
    Copy a file from the target machine to the host machine, via the 
    artifacts mount

    target\_path - path to file in the target host\_path      - path to 
    file on the host machine (e.g. copy test)

\setlength{\parskip}{1ex}
    \end{boxedminipage}

    \label{shutit_global:ShutIt:prompt_cfg}
    \index{shutit\_global \textit{(module)}!shutit\_global.ShutIt \textit{(class)}!shutit\_global.ShutIt.prompt\_cfg \textit{(method)}}

    \vspace{0.5ex}

\hspace{.8\funcindent}\begin{boxedminipage}{\funcwidth}

    \raggedright \textbf{prompt\_cfg}(\textit{self}, \textit{msg}, \textit{sec}, \textit{name}, \textit{ispass}={\tt False})

    \vspace{-1.5ex}

    \rule{\textwidth}{0.5\fboxrule}
\setlength{\parskip}{2ex}
    Prompt for a config value, optionally saving it to the user-level cfg. 
    Only runs if we are in an interactive mode.

    msg    - Message to display to user. sec    - Section of config to add 
    to. name   - Config item name. ispass - Hide the input from the 
    terminal.

\setlength{\parskip}{1ex}
    \end{boxedminipage}

    \label{shutit_global:ShutIt:step_through}
    \index{shutit\_global \textit{(module)}!shutit\_global.ShutIt \textit{(class)}!shutit\_global.ShutIt.step\_through \textit{(method)}}

    \vspace{0.5ex}

\hspace{.8\funcindent}\begin{boxedminipage}{\funcwidth}

    \raggedright \textbf{step\_through}(\textit{self}, \textit{msg}={\tt \texttt{'}\texttt{}\texttt{'}}, \textit{child}={\tt None}, \textit{level}={\tt 1}, \textit{print\_input}={\tt True}, \textit{value}={\tt True})

    \vspace{-1.5ex}

    \rule{\textwidth}{0.5\fboxrule}
\setlength{\parskip}{2ex}
    Implements a step-through function, using pause\_point.

\setlength{\parskip}{1ex}
    \end{boxedminipage}

    \label{shutit_global:ShutIt:pause_point}
    \index{shutit\_global \textit{(module)}!shutit\_global.ShutIt \textit{(class)}!shutit\_global.ShutIt.pause\_point \textit{(method)}}

    \vspace{0.5ex}

\hspace{.8\funcindent}\begin{boxedminipage}{\funcwidth}

    \raggedright \textbf{pause\_point}(\textit{self}, \textit{msg}={\tt \texttt{'}\texttt{}\texttt{'}}, \textit{child}={\tt None}, \textit{print\_input}={\tt True}, \textit{level}={\tt 1}, \textit{resize}={\tt False})

    \vspace{-1.5ex}

    \rule{\textwidth}{0.5\fboxrule}
\setlength{\parskip}{2ex}
    Inserts a pause in the build session, which allows the user to try 
    things out before continuing. Ignored if we are not in an interactive 
    mode, or the interactive level is less than the passed-in one. Designed
    to help debug the build, or drop to on failure so the situation can be 
    debugged.

    \begin{itemize}
    \setlength{\parskip}{0.6ex}
      \item msg         - Message to display to user on pause point.

      \item child       - See send()

      \item print\_input - Whether to take input at this point (ie interact), 
        or simply pause pending any input.

      \item level       - Minimum level to invoke the pause\_point at

    \end{itemize}

\setlength{\parskip}{1ex}
    \end{boxedminipage}

    \label{shutit_global:ShutIt:get_output}
    \index{shutit\_global \textit{(module)}!shutit\_global.ShutIt \textit{(class)}!shutit\_global.ShutIt.get\_output \textit{(method)}}

    \vspace{0.5ex}

\hspace{.8\funcindent}\begin{boxedminipage}{\funcwidth}

    \raggedright \textbf{get\_output}(\textit{self}, \textit{child}={\tt None})

    \vspace{-1.5ex}

    \rule{\textwidth}{0.5\fboxrule}
\setlength{\parskip}{2ex}
    Helper function to get output from latest command run. Use with care - 
    if you are expecting something other than a prompt, this may not return
    what you might expect.

    \begin{itemize}
    \setlength{\parskip}{0.6ex}
      \item child       - See send()

    \end{itemize}

\setlength{\parskip}{1ex}
    \end{boxedminipage}

    \label{shutit_global:ShutIt:get_re_from_child}
    \index{shutit\_global \textit{(module)}!shutit\_global.ShutIt \textit{(class)}!shutit\_global.ShutIt.get\_re\_from\_child \textit{(method)}}

    \vspace{0.5ex}

\hspace{.8\funcindent}\begin{boxedminipage}{\funcwidth}

    \raggedright \textbf{get\_re\_from\_child}(\textit{self}, \textit{string}, \textit{regexp})

    \vspace{-1.5ex}

    \rule{\textwidth}{0.5\fboxrule}
\setlength{\parskip}{2ex}
    Get regular expression from the first of the lines passed in in string 
    that matched.

    Returns None if none of the lines matched.

    Returns True if there are no groups selected in the regexp.

    \begin{itemize}
    \setlength{\parskip}{0.6ex}
      \item string - string to search through lines of

      \item regexp - regexp to search for per line

    \end{itemize}

\setlength{\parskip}{1ex}
    \end{boxedminipage}

    \label{shutit_global:ShutIt:send_and_get_output}
    \index{shutit\_global \textit{(module)}!shutit\_global.ShutIt \textit{(class)}!shutit\_global.ShutIt.send\_and\_get\_output \textit{(method)}}

    \vspace{0.5ex}

\hspace{.8\funcindent}\begin{boxedminipage}{\funcwidth}

    \raggedright \textbf{send\_and\_get\_output}(\textit{self}, \textit{send}, \textit{expect}={\tt None}, \textit{child}={\tt None}, \textit{retry}={\tt 3}, \textit{strip}={\tt True})

    \vspace{-1.5ex}

    \rule{\textwidth}{0.5\fboxrule}
\setlength{\parskip}{2ex}
\begin{alltt}
Returns the output of a command run.
    send() is called, and exit is not checked.

- send   - See send()
- expect - See send()
- child  - See send()
- retry  - Number of times to retry command (default 3)
- strip  - Whether to strip output (defaults to true)
    
\end{alltt}

\setlength{\parskip}{1ex}
    \end{boxedminipage}

    \label{shutit_global:ShutIt:install}
    \index{shutit\_global \textit{(module)}!shutit\_global.ShutIt \textit{(class)}!shutit\_global.ShutIt.install \textit{(method)}}

    \vspace{0.5ex}

\hspace{.8\funcindent}\begin{boxedminipage}{\funcwidth}

    \raggedright \textbf{install}(\textit{self}, \textit{package}, \textit{child}={\tt None}, \textit{expect}={\tt None}, \textit{options}={\tt None}, \textit{timeout}={\tt 3600}, \textit{force}={\tt False}, \textit{check\_exit}={\tt True}, \textit{reinstall}={\tt False})

    \vspace{-1.5ex}

    \rule{\textwidth}{0.5\fboxrule}
\setlength{\parskip}{2ex}
    Distro-independent install function. Takes a package name and runs the 
    relevant install function. Returns true if all ok (ie it's installed), 
    else false.

    \begin{itemize}
    \setlength{\parskip}{0.6ex}
      \item package    - Package to install, which is run through package\_map

      \item expect     - See send()

      \item child      - See send()

      \item timeout    - Timeout to wait for finish of install.

      \item options    - Dictionary for specific options per install tool. 
        Overrides any arguments passed into this function.

      \item force      - force if necessary

      \item check\_exit - If False, failure to install is ok (default True)

      \item reinstall  - Advise a reinstall where possible (default False)

    \end{itemize}

\setlength{\parskip}{1ex}
    \end{boxedminipage}

    \label{shutit_global:ShutIt:remove}
    \index{shutit\_global \textit{(module)}!shutit\_global.ShutIt \textit{(class)}!shutit\_global.ShutIt.remove \textit{(method)}}

    \vspace{0.5ex}

\hspace{.8\funcindent}\begin{boxedminipage}{\funcwidth}

    \raggedright \textbf{remove}(\textit{self}, \textit{package}, \textit{child}={\tt None}, \textit{expect}={\tt None}, \textit{options}={\tt None}, \textit{timeout}={\tt 3600})

    \vspace{-1.5ex}

    \rule{\textwidth}{0.5\fboxrule}
\setlength{\parskip}{2ex}
    Distro-independent remove function. Takes a package name and runs 
    relevant remove function. Returns true if all ok (ie it's installed 
    now), else false.

    \begin{itemize}
    \setlength{\parskip}{0.6ex}
      \item package  - Package to install, which is run through package\_map.

      \item expect   - See send()

      \item child    - See send()

      \item options  - Dict of options to pass to the remove command, mapped by
        install\_type.

      \item timeout  - See send()

    \end{itemize}

\setlength{\parskip}{1ex}
    \end{boxedminipage}

    \label{shutit_global:ShutIt:whoami}
    \index{shutit\_global \textit{(module)}!shutit\_global.ShutIt \textit{(class)}!shutit\_global.ShutIt.whoami \textit{(method)}}

    \vspace{0.5ex}

\hspace{.8\funcindent}\begin{boxedminipage}{\funcwidth}

    \raggedright \textbf{whoami}(\textit{self}, \textit{child}={\tt None}, \textit{expect}={\tt None})

\setlength{\parskip}{2ex}
\setlength{\parskip}{1ex}
    \end{boxedminipage}

    \label{shutit_global:ShutIt:exec_shell}
    \index{shutit\_global \textit{(module)}!shutit\_global.ShutIt \textit{(class)}!shutit\_global.ShutIt.exec\_shell \textit{(method)}}

    \vspace{0.5ex}

\hspace{.8\funcindent}\begin{boxedminipage}{\funcwidth}

    \raggedright \textbf{exec\_shell}(\textit{self}, \textit{command}={\tt \texttt{'}\texttt{bash}\texttt{'}}, \textit{child}={\tt None}, \textit{password}={\tt None})

    \vspace{-1.5ex}

    \rule{\textwidth}{0.5\fboxrule}
\setlength{\parskip}{2ex}
    See login.

    This is the same, except it simply execs a shell, acting like a login. 
    Useful eg if you've just ssh'd in and need to refresh the shell in a 
    simple exec\_shell()/exit\_shell() combo.

    user     - User to login with command  - Command to login with child
    - See send()

\setlength{\parskip}{1ex}
    \end{boxedminipage}

    \label{shutit_global:ShutIt:login_stack_append}
    \index{shutit\_global \textit{(module)}!shutit\_global.ShutIt \textit{(class)}!shutit\_global.ShutIt.login\_stack\_append \textit{(method)}}

    \vspace{0.5ex}

\hspace{.8\funcindent}\begin{boxedminipage}{\funcwidth}

    \raggedright \textbf{login\_stack\_append}(\textit{self}, \textit{r\_id}, \textit{new\_user}={\tt \texttt{'}\texttt{}\texttt{'}})

\setlength{\parskip}{2ex}
\setlength{\parskip}{1ex}
    \end{boxedminipage}

    \label{shutit_global:ShutIt:login}
    \index{shutit\_global \textit{(module)}!shutit\_global.ShutIt \textit{(class)}!shutit\_global.ShutIt.login \textit{(method)}}

    \vspace{0.5ex}

\hspace{.8\funcindent}\begin{boxedminipage}{\funcwidth}

    \raggedright \textbf{login}(\textit{self}, \textit{user}={\tt \texttt{'}\texttt{root}\texttt{'}}, \textit{command}={\tt \texttt{'}\texttt{su -}\texttt{'}}, \textit{child}={\tt None}, \textit{password}={\tt None}, \textit{prompt\_prefix}={\tt None}, \textit{expect}={\tt None}, \textit{timeout}={\tt 20})

    \vspace{-1.5ex}

    \rule{\textwidth}{0.5\fboxrule}
\setlength{\parskip}{2ex}
    Logs the user in with the passed-in password and command. Tracks the 
    login. If used, used logout to log out again. Assumes you are root when
    logging in, so no password required. If not, override the default 
    command for multi-level logins. If passwords are required, see 
    setup\_prompt() and revert\_prompt()

    user      - User to login with command   - Command to login with child
    - See send() prompt\_prefix - Prefix to use in prompt setup

\setlength{\parskip}{1ex}
    \end{boxedminipage}

    \label{shutit_global:ShutIt:logout}
    \index{shutit\_global \textit{(module)}!shutit\_global.ShutIt \textit{(class)}!shutit\_global.ShutIt.logout \textit{(method)}}

    \vspace{0.5ex}

\hspace{.8\funcindent}\begin{boxedminipage}{\funcwidth}

    \raggedright \textbf{logout}(\textit{self}, \textit{child}={\tt None}, \textit{expect}={\tt None}, \textit{command}={\tt \texttt{'}\texttt{exit}\texttt{'}})

    \vspace{-1.5ex}

    \rule{\textwidth}{0.5\fboxrule}
\setlength{\parskip}{2ex}
    Logs the user out. Assumes that login has been called. If login has 
    never been called, throw an error.

    \begin{itemize}
    \setlength{\parskip}{0.6ex}
      \item child              - See send()

      \item expect             - override expect (eg for base\_prompt)

    \end{itemize}

\setlength{\parskip}{1ex}
    \end{boxedminipage}

    \label{shutit_global:ShutIt:logout}
    \index{shutit\_global \textit{(module)}!shutit\_global.ShutIt \textit{(class)}!shutit\_global.ShutIt.logout \textit{(method)}}

    \vspace{0.5ex}

\hspace{.8\funcindent}\begin{boxedminipage}{\funcwidth}

    \raggedright \textbf{exit\_shell}(\textit{self}, \textit{child}={\tt None}, \textit{expect}={\tt None}, \textit{command}={\tt \texttt{'}\texttt{exit}\texttt{'}})

    \vspace{-1.5ex}

    \rule{\textwidth}{0.5\fboxrule}
\setlength{\parskip}{2ex}
    Logs the user out. Assumes that login has been called. If login has 
    never been called, throw an error.

    \begin{itemize}
    \setlength{\parskip}{0.6ex}
      \item child              - See send()

      \item expect             - override expect (eg for base\_prompt)

    \end{itemize}

\setlength{\parskip}{1ex}
    \end{boxedminipage}

    \label{shutit_global:ShutIt:setup_prompt}
    \index{shutit\_global \textit{(module)}!shutit\_global.ShutIt \textit{(class)}!shutit\_global.ShutIt.setup\_prompt \textit{(method)}}

    \vspace{0.5ex}

\hspace{.8\funcindent}\begin{boxedminipage}{\funcwidth}

    \raggedright \textbf{setup\_prompt}(\textit{self}, \textit{prompt\_name}, \textit{prefix}={\tt \texttt{'}\texttt{TMP}\texttt{'}}, \textit{child}={\tt None}, \textit{set\_default\_expect}={\tt True})

    \vspace{-1.5ex}

    \rule{\textwidth}{0.5\fboxrule}
\setlength{\parskip}{2ex}
\begin{alltt}
Use this when you've opened a new shell to set the PS1 to something
sane. By default, it sets up the default expect so you don't have to
worry about it and can just call shutit.send('a command').

If you want simple login and logout, please use login() and logout()
within this module.

Typically it would be used in this boilerplate pattern

shutit.send('su - auser',
            expect=shutit.cfg['expect\_prompts']['base\_prompt'],
            check\_exit=False)
shutit.setup\_prompt('tmp\_prompt')
shutit.send('some command')
[...]
shutit.set\_default\_expect()
shutit.send('exit')

        - prompt\_name        - Reference name for prompt.
        - prefix             - Prompt prefix.
        - child              - See send()
        - set\_default\_expect - Whether to set the default expect to the new prompt.
\end{alltt}

\setlength{\parskip}{1ex}
    \end{boxedminipage}

    \label{shutit_global:ShutIt:revert_prompt}
    \index{shutit\_global \textit{(module)}!shutit\_global.ShutIt \textit{(class)}!shutit\_global.ShutIt.revert\_prompt \textit{(method)}}

    \vspace{0.5ex}

\hspace{.8\funcindent}\begin{boxedminipage}{\funcwidth}

    \raggedright \textbf{revert\_prompt}(\textit{self}, \textit{old\_prompt\_name}, \textit{new\_expect}={\tt None}, \textit{child}={\tt None})

    \vspace{-1.5ex}

    \rule{\textwidth}{0.5\fboxrule}
\setlength{\parskip}{2ex}
    Reverts the prompt to the previous value (passed-in).

    It should be fairly rare to need this. Most of the time you would just 
    exit a subshell rather than resetting the prompt.

    \begin{itemize}
    \setlength{\parskip}{0.6ex}
      \item old\_prompt\_name -

      \item new\_expect      -

      \item child           - See send()

    \end{itemize}

\setlength{\parskip}{1ex}
    \end{boxedminipage}

    \label{shutit_global:ShutIt:get_distro_info}
    \index{shutit\_global \textit{(module)}!shutit\_global.ShutIt \textit{(class)}!shutit\_global.ShutIt.get\_distro\_info \textit{(method)}}

    \vspace{0.5ex}

\hspace{.8\funcindent}\begin{boxedminipage}{\funcwidth}

    \raggedright \textbf{get\_distro\_info}(\textit{self}, \textit{child}={\tt None}, \textit{container}={\tt True})

    \vspace{-1.5ex}

    \rule{\textwidth}{0.5\fboxrule}
\setlength{\parskip}{2ex}
    Get information about which distro we are using.

    Fails if distro could not be determined. Should be called with the 
    container is started up, and uses as core info as possible.

    \begin{itemize}
    \setlength{\parskip}{0.6ex}
      \item child              - See send()

      \item container          - If True, we are in the container shell, 
        otherwise we are gathering info about another shell

    \end{itemize}

\setlength{\parskip}{1ex}
    \end{boxedminipage}

    \label{shutit_global:ShutIt:lsb_release}
    \index{shutit\_global \textit{(module)}!shutit\_global.ShutIt \textit{(class)}!shutit\_global.ShutIt.lsb\_release \textit{(method)}}

    \vspace{0.5ex}

\hspace{.8\funcindent}\begin{boxedminipage}{\funcwidth}

    \raggedright \textbf{lsb\_release}(\textit{self}, \textit{child}={\tt None})

\setlength{\parskip}{2ex}
\setlength{\parskip}{1ex}
    \end{boxedminipage}

    \label{shutit_global:ShutIt:set_password}
    \index{shutit\_global \textit{(module)}!shutit\_global.ShutIt \textit{(class)}!shutit\_global.ShutIt.set\_password \textit{(method)}}

    \vspace{0.5ex}

\hspace{.8\funcindent}\begin{boxedminipage}{\funcwidth}

    \raggedright \textbf{set\_password}(\textit{self}, \textit{password}, \textit{user}={\tt \texttt{'}\texttt{}\texttt{'}}, \textit{child}={\tt None}, \textit{expect}={\tt None})

    \vspace{-1.5ex}

    \rule{\textwidth}{0.5\fboxrule}
\setlength{\parskip}{2ex}
\begin{alltt}
Sets the password for the current user or passed-in user.

- password - 
- user     - 
- expect   - See send()
- child    - See send()
\end{alltt}

\setlength{\parskip}{1ex}
    \end{boxedminipage}

    \label{shutit_global:ShutIt:is_user_id_available}
    \index{shutit\_global \textit{(module)}!shutit\_global.ShutIt \textit{(class)}!shutit\_global.ShutIt.is\_user\_id\_available \textit{(method)}}

    \vspace{0.5ex}

\hspace{.8\funcindent}\begin{boxedminipage}{\funcwidth}

    \raggedright \textbf{is\_user\_id\_available}(\textit{self}, \textit{user\_id}, \textit{child}={\tt None}, \textit{expect}={\tt None})

    \vspace{-1.5ex}

    \rule{\textwidth}{0.5\fboxrule}
\setlength{\parskip}{2ex}
\begin{alltt}
Determine whether a user\_id for a user is available.

- user\_id  - 
- expect   - See send()
- child    - See send()
\end{alltt}

\setlength{\parskip}{1ex}
    \end{boxedminipage}

    \label{shutit_global:ShutIt:push_repository}
    \index{shutit\_global \textit{(module)}!shutit\_global.ShutIt \textit{(class)}!shutit\_global.ShutIt.push\_repository \textit{(method)}}

    \vspace{0.5ex}

\hspace{.8\funcindent}\begin{boxedminipage}{\funcwidth}

    \raggedright \textbf{push\_repository}(\textit{self}, \textit{repository}, \textit{docker\_executable}={\tt \texttt{'}\texttt{docker}\texttt{'}}, \textit{child}={\tt None}, \textit{expect}={\tt None})

    \vspace{-1.5ex}

    \rule{\textwidth}{0.5\fboxrule}
\setlength{\parskip}{2ex}
\begin{alltt}
Pushes the repository.

- repository        - 
- docker\_executable -
- expect            - See send()
- child             - See send()
\end{alltt}

\setlength{\parskip}{1ex}
    \end{boxedminipage}

    \label{shutit_global:ShutIt:do_repository_work}
    \index{shutit\_global \textit{(module)}!shutit\_global.ShutIt \textit{(class)}!shutit\_global.ShutIt.do\_repository\_work \textit{(method)}}

    \vspace{0.5ex}

\hspace{.8\funcindent}\begin{boxedminipage}{\funcwidth}

    \raggedright \textbf{do\_repository\_work}(\textit{self}, \textit{repo\_name}, \textit{repo\_tag}={\tt None}, \textit{expect}={\tt None}, \textit{docker\_executable}={\tt \texttt{'}\texttt{docker.io}\texttt{'}}, \textit{password}={\tt None}, \textit{force}={\tt None})

    \vspace{-1.5ex}

    \rule{\textwidth}{0.5\fboxrule}
\setlength{\parskip}{2ex}
    Commit, tag, push, tar a docker container based on the configuration we
    have.

    \begin{itemize}
    \setlength{\parskip}{0.6ex}
      \item repo\_name         -

      \item expect            - See send()

      \item docker\_executable -

      \item password          -

      \item force             -

    \end{itemize}

\setlength{\parskip}{1ex}
    \end{boxedminipage}

    \label{shutit_global:ShutIt:get_config}
    \index{shutit\_global \textit{(module)}!shutit\_global.ShutIt \textit{(class)}!shutit\_global.ShutIt.get\_config \textit{(method)}}

    \vspace{0.5ex}

\hspace{.8\funcindent}\begin{boxedminipage}{\funcwidth}

    \raggedright \textbf{get\_config}(\textit{self}, \textit{module\_id}, \textit{option}, \textit{default}={\tt None}, \textit{boolean}={\tt False}, \textit{forcedefault}={\tt False}, \textit{forcenone}={\tt False}, \textit{hint}={\tt None})

    \vspace{-1.5ex}

    \rule{\textwidth}{0.5\fboxrule}
\setlength{\parskip}{2ex}
    Gets a specific config from the config files, allowing for a default.

    Handles booleans vs strings appropriately.

    module\_id    - module id this relates to, eg 
    com.mycorp.mymodule.mymodule option       - config item to set default
    - default value if not set in files boolean      - whether this is a 
    boolean value or not (default False) forcedefault - if set to true, 
    allows you to override any value already set (default False) forcenone
    - if set to true, allows you to set the value to None (default False) 
    hint         - if we are interactive, then show this prompt to help the
    user input a useful value

\setlength{\parskip}{1ex}
    \end{boxedminipage}

    \label{shutit_global:ShutIt:get_ip_address}
    \index{shutit\_global \textit{(module)}!shutit\_global.ShutIt \textit{(class)}!shutit\_global.ShutIt.get\_ip\_address \textit{(method)}}

    \vspace{0.5ex}

\hspace{.8\funcindent}\begin{boxedminipage}{\funcwidth}

    \raggedright \textbf{get\_ip\_address}(\textit{self}, \textit{ip\_family}={\tt \texttt{'}\texttt{4}\texttt{'}}, \textit{ip\_object}={\tt \texttt{'}\texttt{addr}\texttt{'}}, \textit{command}={\tt \texttt{'}\texttt{ip}\texttt{'}}, \textit{interface}={\tt \texttt{'}\texttt{eth0}\texttt{'}})

    \vspace{-1.5ex}

    \rule{\textwidth}{0.5\fboxrule}
\setlength{\parskip}{2ex}
    Gets the ip address based on the args given. Assumes command exists. 
    ip\_family - type of ip family, defaults to 4 ip\_object - type of ip 
    object, defaults to "addr" command   - defaults to "ip" interface - 
    defaults to "eth0"

\setlength{\parskip}{1ex}
    \end{boxedminipage}

    \label{shutit_global:ShutIt:record_config}
    \index{shutit\_global \textit{(module)}!shutit\_global.ShutIt \textit{(class)}!shutit\_global.ShutIt.record\_config \textit{(method)}}

    \vspace{0.5ex}

\hspace{.8\funcindent}\begin{boxedminipage}{\funcwidth}

    \raggedright \textbf{record\_config}(\textit{self})

    \vspace{-1.5ex}

    \rule{\textwidth}{0.5\fboxrule}
\setlength{\parskip}{2ex}
    Put the config in a file in the target.

\setlength{\parskip}{1ex}
    \end{boxedminipage}

    \label{shutit_global:ShutIt:get_emailer}
    \index{shutit\_global \textit{(module)}!shutit\_global.ShutIt \textit{(class)}!shutit\_global.ShutIt.get\_emailer \textit{(method)}}

    \vspace{0.5ex}

\hspace{.8\funcindent}\begin{boxedminipage}{\funcwidth}

    \raggedright \textbf{get\_emailer}(\textit{self}, \textit{cfg\_section})

    \vspace{-1.5ex}

    \rule{\textwidth}{0.5\fboxrule}
\setlength{\parskip}{2ex}
    Sends an email using the mailer

\setlength{\parskip}{1ex}
    \end{boxedminipage}


\large{\textbf{\textit{Inherited from object}}}

\begin{quote}
\_\_delattr\_\_(), \_\_format\_\_(), \_\_getattribute\_\_(), \_\_hash\_\_(), \_\_new\_\_(), \_\_reduce\_\_(), \_\_reduce\_ex\_\_(), \_\_repr\_\_(), \_\_setattr\_\_(), \_\_sizeof\_\_(), \_\_str\_\_(), \_\_subclasshook\_\_()
\end{quote}

%%%%%%%%%%%%%%%%%%%%%%%%%%%%%%%%%%%%%%%%%%%%%%%%%%%%%%%%%%%%%%%%%%%%%%%%%%%
%%                              Properties                               %%
%%%%%%%%%%%%%%%%%%%%%%%%%%%%%%%%%%%%%%%%%%%%%%%%%%%%%%%%%%%%%%%%%%%%%%%%%%%

  \subsubsection{Properties}

    \vspace{-1cm}
\hspace{\varindent}\begin{longtable}{|p{\varnamewidth}|p{\vardescrwidth}|l}
\cline{1-2}
\cline{1-2} \centering \textbf{Name} & \centering \textbf{Description}& \\
\cline{1-2}
\endhead\cline{1-2}\multicolumn{3}{r}{\small\textit{continued on next page}}\\\endfoot\cline{1-2}
\endlastfoot\multicolumn{2}{|l|}{\textit{Inherited from object}}\\
\multicolumn{2}{|p{\varwidth}|}{\raggedright \_\_class\_\_}\\
\cline{1-2}
\end{longtable}

    \index{shutit\_global \textit{(module)}!shutit\_global.ShutIt \textit{(class)}|)}
    \index{shutit\_global \textit{(module)}|)}
